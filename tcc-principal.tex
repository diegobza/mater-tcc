\documentclass{article}
\usepackage[utf8]{inputenc}
\usepackage[T1]{fontenc}
\usepackage[brazil]{babel}         % Português brasileiro.
\usepackage{sectsty}               % Mudança de estilo das seções.
\usepackage{indentfirst}           % Espaçamento de primeiro parágrafo.
\usepackage{setspace}              % Espaçamento entrelinhas.
\usepackage[hang]{footmisc}        % Espaçamento da nota de rodapé.
\usepackage{fancyhdr}              % Cabeçalho para numeração de páginas.
\usepackage{geometry}
\setlength\headheight{3.35mm}
\setlength\headsep{6.65mm}
\geometry{
  a4paper,
  left=3cm,
  right=2cm,
  top=3cm,
  bottom=2cm,
}
\setlength\parindent{2cm}          % Espaçamento de primeira linha.
\setlength\footnotemargin{10pt}    % Espaçamento esquerdo da nota de rodapé.
\allsectionsfont{\normalsize}      % Tamanho da fonte 12 para seções.
\pagestyle{fancy}                  % Cabeçalho e rodapé vazios.
\fancyhf{}                         % Apaga todos os campos do cabeçalho e rodapé.
\rhead{\footnotesize\thepage}
\renewcommand{\headrulewidth}{0pt} % Remove linha do cabeçalho.

\begin{document}
  \newcommand{\TITULOP}{TÍTULO DO ARTIGO EM PORTUGUÊS}
%\newcommand{\SUBTITULOP}{SUBTÍTULO EM PORTUGUÊS}
\newcommand{\TITULOE}{TÍTULO DO ARTIGO EM LÍNGUA ESTRANGEIRA}
%\newcommand{\SUBTITULOE}{SUBTÍTULO EM LÍNGUA ESTRANGEIRA}
\newcommand{\AUTOR}{Nome do Aluno}
\newcommand{\AUTORCURRICULO}{Discente do Curso de Graduação em Xxxxxx da
Faculdade de Ciências e Tecnologia Mater Christi. E-mail: xxxxxx@xxxxx.com.br.}
\newcommand{\ORIENTADOR}{Nome do Professor Orientador}
\newcommand{\ORIENTADORCURRICULO}{Docente do Curso de Graduação em Xxxxxx da
Faculdade de Ciências e Tecnologia Mater Christi. E-mail: xxxxxx@xxxxx.com.br.
URL do Currículo Lattes: http://lattes.cnpq.br/00000000.}

\begin{onehalfspace}
  \begin{center}
    \ifdefined\SUBTITULOP
      \textbf{\TITULOP: \SUBTITULOP}
    \else
      \textbf{\TITULOP}
    \fi

    \bigskip

    \ifdefined\SUBTITULOE
      \textbf{\TITULOE: \SUBTITULOE}
    \else
      \textbf{\TITULOE}
    \fi
  \end{center}

  \bigskip

  \begin{flushright}
    \AUTOR\footnote{\AUTORCURRICULO}

    \ORIENTADOR\footnote{\ORIENTADORCURRICULO}
  \end{flushright}
\end{onehalfspace}

  \begin{center}
  \textbf{RESUMO}
\end{center}

  \section{INTRODUÇÃO}

\onehalfspacing A introdução é a parte inicial do artigo, que contém informações objetivas para
situar o tema do trabalho. Deve constar a delimitação do assunto estudado, a
justificativa que levou à escolha do tema, a pertinência e relevância do tema
na atualidade, a problemática do estudo e os objetivos do estudo. Na introdução
não deve constar ilustrações (tabela, quadros e figura) e referencial teórico.

  \section{CONCLUSÃO}

Seção em que se descrevem as informações finais do trabalho e mostra a
relação das hipóteses levantadas com os objetivos do estudo que partem da
problemática do tema em questão. Compreende uma abordagem descritiva e
informativa, nas quais são apresentados os principais resultados encontrados. É
esperado também que seja realizada uma autocrítica em relação ao tema. As
conclusões devem responder às questões da pesquisa, correspondentes aos
objetivos. As recomendações e sugestões devem prospectar trabalhos futuros.

\end{document}
